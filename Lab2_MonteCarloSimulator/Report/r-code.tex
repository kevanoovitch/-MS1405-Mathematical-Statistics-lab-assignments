\begin{minted}[breaklines, linenos, frame=single, bgcolor=lightgray]{r}
    library(moments)
    library(ggplot2)
    library(gridExtra)

    Log_Rayleigh_likelihood= function(sd, mu) {
        n = length(sd) # Number of observations
        result = n*log(pi) + sum(log(sd)) - n*log(2) - 2*n*log(mu) - (pi* sum(sd^2))/(4*mu^2) 
        return(result)
    }

    numericalDerivative = function(f, data){
        # Ensure data is numeric
        if (!is.numeric(data)) {
            stop("Error: Data is not numeric.")
        }

        result = optim(
        par = 5,
        fn = function(mu) f(data, mu),  # Optimize over mu, passing data via closure 
        method = "L-BFGS-B", # BFGS with restrictions
        lower = 0.001,  # Set a lower bound to avoid non-positive mu values
        control = list(fnscale= -1))

        return(result)
    }

    analyticalDerivative = function(mu, y){
        n = length(y)
        (-2*n)/mu + (pi * sum(y^2))/(2*mu^3)
    }

    analyticalMaximizing = function(y) {
        result = optim(par = 5, 
                       fn = function(mu) Log_Rayleigh_likelihood(y, mu), # Objective function
                       gr = function(mu) analyticalDerivative(mu, y),    # Gradient function
                       method = "L-BFGS-B", # BFGS with restrictions
                       lower = 0.001,  # Set a lower bound to avoid non-positive mu values
                       control = list(fnscale = -1))
        return(result)
    }

    rr = function(mu, n) {
        u = runif(n)
        r_samples = 2 * mu * sqrt(-log(1 - u) / pi)
        return(r_samples)
    }

    calc_bias = function(estimates, true_value){
        return(mean(estimates) - true_value)
    }

    calc_mse = function(estimates, true_value){
        return(mean((estimates) - true_value)^2)
    }

    calculate_metrics = function(estimates, true_value){
        mean_estimate = mean(estimates)
        bias = calc_bias(estimates, true_value)
        mse = calc_mse(estimates, true_value)
        skewness_val = skewness(estimates)
        kurtosis_val = kurtosis(estimates) 

        return(list(mean = mean_estimate, bias = bias, mse = mse, skewness = skewness_val, kurtosis = kurtosis_val))
    }

    computation = function(n) {
        r_values = rr(mu = 5, n) # Generate random samples with n in size from n_values
        log_likelihood = Log_Rayleigh_likelihood(r_values, mu = 5) # mu is an initial guess

        numerical_result = numericalDerivative(Log_Rayleigh_likelihood, r_values)
        analytical_maximization = analyticalMaximizing(r_values)

        return(list(
            numerical_result = numerical_result, 
            analytical_maximization = analytical_maximization
        ))
    }

    clean_results = function(result, true_mu, method_type) {
        converged_estimates = list()

        # Filter out only converged results
        res = result[[method_type]]

        # Check if 'res' is a list and contains 'convergence'
        if (is.list(res) && !is.null(res[["convergence"]]) && res[["convergence"]] == 0) {
            converged_estimates = res$par  # Store the estimated parameter
        }

        if (length(converged_estimates) > 0) {
            # Calculate performance metrics
            metrics = calculate_metrics(converged_estimates, true_mu)

            return(list(
                converged_estimates = converged_estimates,
                metrics = metrics
            ))
        } else {
            return ("No converged models to analyze")
        }
    }

    # --- Plot Functions --- #
    plot_mean_estimates <- function(estimate_df_numerical, estimate_df_analytical, true_mu) {

        # Calculate the mean estimate for each sample size in numerical method
        mean_numerical <- aggregate(Estimate ~ Sample_Size, data = estimate_df_numerical, FUN = mean)
        mean_numerical$Method <- "Numerical"

        # Calculate the mean estimate for each sample size in analytical method
        mean_analytical <- aggregate(Estimate ~ Sample_Size, data = estimate_df_analytical, FUN = mean)
        mean_analytical$Method <- "Analytical"

        # Combine both data frames for plotting
        combined_mean_df <- rbind(mean_numerical, mean_analytical)

        # Plot mean estimates with a horizontal line at true_mu
        ggplot(combined_mean_df, aes(x = Sample_Size, y = Estimate, color = Method, group = Method)) +
            geom_line(aes(linetype = Method), linewidth = 1.2) +  # Use linewidth instead of size
            geom_point(size = 3) +
            geom_hline(yintercept = true_mu, linetype = "dashed", color = "black", linewidth = 1) +
            labs(title = "Mean Estimates of Numerical and Analytical Methods",
                 x = "Sample Size", y = "Mean Estimate") +
            theme_minimal() +
            scale_color_manual(values = c("Numerical" = "blue", "Analytical" = "red")) +
            theme(legend.position = "top")
    }

    # Function to plot the bell curves
    plot_bell_curve <- function(numerical_estimates, analytical_estimates, true_mu) {
        # Convert estimates to data frames for plotting
        df_numerical <- data.frame(x = numerical_estimates, Method = "Numerical")
        df_analytical <- data.frame(x = analytical_estimates, Method = "Analytical")

        # Combine the two data frames
        df_combined <- rbind(df_numerical, df_analytical)

        # Get standard deviation for the normal distribution based on estimates
        combined_sd <- sd(df_combined$x)

        # Create the normal distribution curve using the true_mu and calculated std deviation
        normal_curve <- data.frame(
            x = seq(min(df_combined$x) - 1, max(df_combined$x) + 1, length.out = 100),
            y = dnorm(seq(min(df_combined$x) - 1, max(df_combined$x) + 1, length.out = 100),
                      mean = true_mu,
                      sd = combined_sd),
            Method = "Normal"
        )

        # Scale the normal curve's density to match the density of estimates
        scale_factor <- max(density(numerical_estimates)$y, density(analytical_estimates)$y) / max(normal_curve$y)
        normal_curve$y <- normal_curve$y * scale_factor

        # Plot the curves
        ggplot() +
            geom_density(data = df_combined, aes(x = x, color = Method, linetype = Method), size = 1.5) +   # Plot density for numerical and analytical estimates
            geom_line(data = normal_curve, aes(x = x, y = y, color = Method, linetype = Method), size = 1.5) +  # Add the normal distribution curve
            labs(title = "Bell Curves for Numerical, Analytical, and Normal Distributions",
                 x = "Estimate", y = "Density") +
            theme_minimal() +
            scale_color_manual(values = c("Numerical" = "blue", "Analytical" = "red", "Normal" = "black")) +
            scale_linetype_manual(values = c("Numerical" = "dashed", "Analytical" = "solid", "Normal" = "dotted")) +
            theme(legend.position = "top")
    }





    # --- Main Computation --- #
    set.seed(123) # For reproducibility in different runs of the code
    n_value = c(10, 20, 50, 90, 140)
    true_mu = 5

    # Separate lists for the results
    all_results_numerical = list() 
    all_results_analytical = list()

    # Run the Monte Carlo simulation and store results
    bias_mse_df_numerical = data.frame(Sample_Size = integer(), Bias = double(), MSE = double())
    bias_mse_df_analytical = data.frame(Sample_Size = integer(), Bias = double(), MSE = double())

    estimate_df_numerical = data.frame(Sample_Size = integer(), Estimate = double(), Method = character())
    estimate_df_analytical = data.frame(Sample_Size = integer(), Estimate = double(), Method = character())

    # --- Sample Sizes Loop --- #
    n_reps = 1000

    # Iterate through all given sample sizes
    for (n in n_value) {
        cat("\n--- Processing n =", n, "---\n")

        numerical_estimates = c()
        analytical_estimates = c()

        # Repeat the computation 1000 times for each sample size
        for (rep in 1:n_reps) {
            result <- computation(n)

            # Clean numerical and analytical results separately
            cleaned_numerical = clean_results(result, true_mu, "numerical_result")
            cleaned_analytical = clean_results(result, true_mu, "analytical_maximization")

            # Store numerical results if they exist
            if (is.list(cleaned_numerical)) {
                numerical_estimates = c(numerical_estimates, cleaned_numerical$converged_estimates)
            }

            # Store analytical results if they exist
            if (is.list(cleaned_analytical)) {
                analytical_estimates = c(analytical_estimates, cleaned_analytical$converged_estimates)
            }
        }

        # After 1000 repetitions
        if (length(numerical_estimates) > 0) {
            cat("\nNumerical Method Results for n =", n, " after", n_reps, "repetitions:\n")
            numerical_metrics = calculate_metrics(numerical_estimates, true_mu)
            print(numerical_metrics)

            if (is.list(cleaned_numerical)) {
                estimate_df_numerical = rbind(estimate_df_numerical, data.frame(
                    Sample_Size = rep(n, length(cleaned_numerical$converged_estimates)),
                    Estimate = numerical_estimates,
                    Method = "Numerical"
                ))
            }
        }

        if (length(analytical_estimates) > 0) {
            cat("\nAnalytical Method Results for n =", n, " after", n_reps, "repetitions:\n")
            analytical_metrics = calculate_metrics(analytical_estimates, true_mu)
            print(analytical_metrics)

            if (is.list(cleaned_analytical)) {
                estimate_df_analytical = rbind(estimate_df_analytical, data.frame(
                    Sample_Size = rep(n, length(cleaned_analytical$converged_estimates)),
                    Estimate = analytical_estimates,
                    Method = "Analytical"
                ))
            }
        }

    }

    # After the simulations
    # Call the function to plot the mean estimates
    print(plot_mean_estimates(estimate_df_numerical, estimate_df_analytical, true_mu = 5))
    print(plot_bell_curve(numerical_estimates, analytical_estimates, true_mu = 5))
\end{minted}

% ----- Results -----