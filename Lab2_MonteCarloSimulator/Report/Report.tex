\documentclass{report}
\usepackage[utf8]{inputenc}   % Encoding support
\usepackage{graphicx}         % To include images
\usepackage{listings}         % For code formatting
\usepackage{xcolor}           % To add colors
\usepackage{hyperref}         % For clickable links and references
\usepackage{amsmath, amssymb,amsthm} %For math
\usepackage{comment}
\usepackage{float}
\usepackage{minted}


\title{Report II}
\author{Kevin Deshayes}
\date{\2024-10-14}

\begin{document}

% ----- Title Page -----
\begin{titlepage}
    \centering
    {\huge\bfseries Report on Lab 2 \\[1cm]}  % Title
    \textbf{Author:} Kevin Deshayes\\[0.5cm]  % Author Name
    \textbf{Email:} kede23@student.bth.se\\[0.5cm]  % Email
    \textbf{Course Name:} Mathematical Statistics \\[0.5cm]  % Course Name
    \textbf{Course Code:} MS1403\\[1.5cm]  % Course Code
    \includegraphics[width=0.6\textwidth]{Images/BTH_logo_gray} % Placeholder image
    \vfill
    \textbf{Date:} October 14, 2024  % Manually add the date
    \vspace{2cm}
\end{titlepage}

% ----- Table of Contents -----
\tableofcontents
\newpage
% ----- Code Example -----
\section{R code}\label{sec:r-code}
\begin{minted}[breaklines, linenos, frame=single, bgcolor=lightgray]{r}
    library(moments)
    library(ggplot2)
    library(gridExtra)

    Log_Rayleigh_likelihood= function(sd, mu) {
        n = length(sd) # Number of observations
        result = n*log(pi) + sum(log(sd)) - n*log(2) - 2*n*log(mu) - (pi* sum(sd^2))/(4*mu^2) 
        return(result)
    }

    numericalDerivative = function(f, data){
        # Ensure data is numeric
        if (!is.numeric(data)) {
            stop("Error: Data is not numeric.")
        }

        result = optim(
        par = 5,
        fn = function(mu) f(data, mu),  # Optimize over mu, passing data via closure 
        method = "L-BFGS-B", # BFGS with restrictions
        lower = 0.001,  # Set a lower bound to avoid non-positive mu values
        control = list(fnscale= -1))

        return(result)
    }

    analyticalDerivative = function(mu, y){
        n = length(y)
        (-2*n)/mu + (pi * sum(y^2))/(2*mu^3)
    }

    analyticalMaximizing = function(y) {
        result = optim(par = 5, 
                       fn = function(mu) Log_Rayleigh_likelihood(y, mu), # Objective function
                       gr = function(mu) analyticalDerivative(mu, y),    # Gradient function
                       method = "L-BFGS-B", # BFGS with restrictions
                       lower = 0.001,  # Set a lower bound to avoid non-positive mu values
                       control = list(fnscale = -1))
        return(result)
    }

    rr = function(mu, n) {
        u = runif(n)
        r_samples = 2 * mu * sqrt(-log(1 - u) / pi)
        return(r_samples)
    }

    calc_bias = function(estimates, true_value){
        return(mean(estimates) - true_value)
    }

    calc_mse = function(estimates, true_value){
        return(mean((estimates) - true_value)^2)
    }

    calculate_metrics = function(estimates, true_value){
        mean_estimate = mean(estimates)
        bias = calc_bias(estimates, true_value)
        mse = calc_mse(estimates, true_value)
        skewness_val = skewness(estimates)
        kurtosis_val = kurtosis(estimates) 

        return(list(mean = mean_estimate, bias = bias, mse = mse, skewness = skewness_val, kurtosis = kurtosis_val))
    }

    computation = function(n) {
        r_values = rr(mu = 5, n) # Generate random samples with n in size from n_values
        log_likelihood = Log_Rayleigh_likelihood(r_values, mu = 5) # mu is an initial guess

        numerical_result = numericalDerivative(Log_Rayleigh_likelihood, r_values)
        analytical_maximization = analyticalMaximizing(r_values)

        return(list(
            numerical_result = numerical_result, 
            analytical_maximization = analytical_maximization
        ))
    }

    clean_results = function(result, true_mu, method_type) {
        converged_estimates = list()

        # Filter out only converged results
        res = result[[method_type]]

        # Check if 'res' is a list and contains 'convergence'
        if (is.list(res) && !is.null(res[["convergence"]]) && res[["convergence"]] == 0) {
            converged_estimates = res$par  # Store the estimated parameter
        }

        if (length(converged_estimates) > 0) {
            # Calculate performance metrics
            metrics = calculate_metrics(converged_estimates, true_mu)

            return(list(
                converged_estimates = converged_estimates,
                metrics = metrics
            ))
        } else {
            return ("No converged models to analyze")
        }
    }

    # --- Plot Functions --- #
    plot_mean_estimates <- function(estimate_df_numerical, estimate_df_analytical, true_mu) {

        # Calculate the mean estimate for each sample size in numerical method
        mean_numerical <- aggregate(Estimate ~ Sample_Size, data = estimate_df_numerical, FUN = mean)
        mean_numerical$Method <- "Numerical"

        # Calculate the mean estimate for each sample size in analytical method
        mean_analytical <- aggregate(Estimate ~ Sample_Size, data = estimate_df_analytical, FUN = mean)
        mean_analytical$Method <- "Analytical"

        # Combine both data frames for plotting
        combined_mean_df <- rbind(mean_numerical, mean_analytical)

        # Plot mean estimates with a horizontal line at true_mu
        ggplot(combined_mean_df, aes(x = Sample_Size, y = Estimate, color = Method, group = Method)) +
            geom_line(aes(linetype = Method), linewidth = 1.2) +  # Use linewidth instead of size
            geom_point(size = 3) +
            geom_hline(yintercept = true_mu, linetype = "dashed", color = "black", linewidth = 1) +
            labs(title = "Mean Estimates of Numerical and Analytical Methods",
                 x = "Sample Size", y = "Mean Estimate") +
            theme_minimal() +
            scale_color_manual(values = c("Numerical" = "blue", "Analytical" = "red")) +
            theme(legend.position = "top")
    }

    # Function to plot the bell curves
    plot_bell_curve <- function(numerical_estimates, analytical_estimates, true_mu) {
        # Convert estimates to data frames for plotting
        df_numerical <- data.frame(x = numerical_estimates, Method = "Numerical")
        df_analytical <- data.frame(x = analytical_estimates, Method = "Analytical")

        # Combine the two data frames
        df_combined <- rbind(df_numerical, df_analytical)

        # Get standard deviation for the normal distribution based on estimates
        combined_sd <- sd(df_combined$x)

        # Create the normal distribution curve using the true_mu and calculated std deviation
        normal_curve <- data.frame(
            x = seq(min(df_combined$x) - 1, max(df_combined$x) + 1, length.out = 100),
            y = dnorm(seq(min(df_combined$x) - 1, max(df_combined$x) + 1, length.out = 100),
                      mean = true_mu,
                      sd = combined_sd),
            Method = "Normal"
        )

        # Scale the normal curve's density to match the density of estimates
        scale_factor <- max(density(numerical_estimates)$y, density(analytical_estimates)$y) / max(normal_curve$y)
        normal_curve$y <- normal_curve$y * scale_factor

        # Plot the curves
        ggplot() +
            geom_density(data = df_combined, aes(x = x, color = Method, linetype = Method), size = 1.5) +   # Plot density for numerical and analytical estimates
            geom_line(data = normal_curve, aes(x = x, y = y, color = Method, linetype = Method), size = 1.5) +  # Add the normal distribution curve
            labs(title = "Bell Curves for Numerical, Analytical, and Normal Distributions",
                 x = "Estimate", y = "Density") +
            theme_minimal() +
            scale_color_manual(values = c("Numerical" = "blue", "Analytical" = "red", "Normal" = "black")) +
            scale_linetype_manual(values = c("Numerical" = "dashed", "Analytical" = "solid", "Normal" = "dotted")) +
            theme(legend.position = "top")
    }





    # --- Main Computation --- #
    set.seed(123) # For reproducibility in different runs of the code
    n_value = c(10, 20, 50, 90, 140)
    true_mu = 5

    # Separate lists for the results
    all_results_numerical = list() 
    all_results_analytical = list()

    # Run the Monte Carlo simulation and store results
    bias_mse_df_numerical = data.frame(Sample_Size = integer(), Bias = double(), MSE = double())
    bias_mse_df_analytical = data.frame(Sample_Size = integer(), Bias = double(), MSE = double())

    estimate_df_numerical = data.frame(Sample_Size = integer(), Estimate = double(), Method = character())
    estimate_df_analytical = data.frame(Sample_Size = integer(), Estimate = double(), Method = character())

    # --- Sample Sizes Loop --- #
    n_reps = 1000

    # Iterate through all given sample sizes
    for (n in n_value) {
        cat("\n--- Processing n =", n, "---\n")

        numerical_estimates = c()
        analytical_estimates = c()

        # Repeat the computation 1000 times for each sample size
        for (rep in 1:n_reps) {
            result <- computation(n)

            # Clean numerical and analytical results separately
            cleaned_numerical = clean_results(result, true_mu, "numerical_result")
            cleaned_analytical = clean_results(result, true_mu, "analytical_maximization")

            # Store numerical results if they exist
            if (is.list(cleaned_numerical)) {
                numerical_estimates = c(numerical_estimates, cleaned_numerical$converged_estimates)
            }

            # Store analytical results if they exist
            if (is.list(cleaned_analytical)) {
                analytical_estimates = c(analytical_estimates, cleaned_analytical$converged_estimates)
            }
        }

        # After 1000 repetitions
        if (length(numerical_estimates) > 0) {
            cat("\nNumerical Method Results for n =", n, " after", n_reps, "repetitions:\n")
            numerical_metrics = calculate_metrics(numerical_estimates, true_mu)
            print(numerical_metrics)

            if (is.list(cleaned_numerical)) {
                estimate_df_numerical = rbind(estimate_df_numerical, data.frame(
                    Sample_Size = rep(n, length(cleaned_numerical$converged_estimates)),
                    Estimate = numerical_estimates,
                    Method = "Numerical"
                ))
            }
        }

        if (length(analytical_estimates) > 0) {
            cat("\nAnalytical Method Results for n =", n, " after", n_reps, "repetitions:\n")
            analytical_metrics = calculate_metrics(analytical_estimates, true_mu)
            print(analytical_metrics)

            if (is.list(cleaned_analytical)) {
                estimate_df_analytical = rbind(estimate_df_analytical, data.frame(
                    Sample_Size = rep(n, length(cleaned_analytical$converged_estimates)),
                    Estimate = analytical_estimates,
                    Method = "Analytical"
                ))
            }
        }

    }

    # After the simulations
    # Call the function to plot the mean estimates
    print(plot_mean_estimates(estimate_df_numerical, estimate_df_analytical, true_mu = 5))
    print(plot_bell_curve(numerical_estimates, analytical_estimates, true_mu = 5))
\end{minted}

% ----- Results -----

\section{Results}\label{sec:results}
\begin{table}[h]
    \centering
    \begin{tabular}{| c | c | c | c | c | c | c |}
        \hline
        $n$ & Method     & Mean        & Bias         & MSE                     & Skewness    & Kurtosis    \\
        \hline
        10  & Numerical  & 4.905042000 & -0.094957570 & 0.009016940             & 0.240844800 & 3.216145000 \\
        10  & Analytical & 4.905042000 & -0.094957740 & 0.009016973             & 0.240844800 & 3.216145000 \\
        \hline
        20  & Numerical  & 4.950229000 & -0.049771060 & 0.002477158             & 0.124840100 & 3.030776000 \\
        20  & Analytical & 4.950229000 & -0.049771230 & 0.002477175             & 0.124840100 & 3.030776000 \\
        \hline
        50  & Numerical  & 4.984634000 & -0.015365770 & 0.000236106             & 0.061915120 & 3.024987000 \\
        50  & Analytical & 4.984634000 & -0.015365940 & 0.000236112             & 0.061915110 & 3.024987000 \\
        \hline
        90  & Numerical  & 5.007175000 & 0.007175166  & $5.1483 \times 10^{-5}$ & 0.058607470 & 3.049959000 \\
        90  & Analytical & 5.007175000 & 0.007175000  & $5.1481 \times 10^{-5}$ & 0.058607470 & 3.049959000 \\
        \hline
        140 & Numerical  & 4.995738000 & -0.004261530 & $1.8161 \times 10^{-5}$ & 0.085037030 & 3.152488000 \\
        140 & Analytical & 4.995738000 & -0.004261697 & $1.8162 \times 10^{-5}$ & 0.085037020 & 3.152488000 \\
        \hline
    \end{tabular}
    \caption{Comparison of Numerical and Analytical Methods for different $n$ after 1000 repetitions}\label{tab:table}
\end{table}

\begin{figure}[H]
    \centering
    \includegraphics[width=1\linewidth]{Images/EstimationCompreSeeded}
    \caption{Comparison on estimations}
    \label{fig:Estimations}
\end{figure}

\begin{figure}[H]
    \centering
    \includegraphics[width=1\linewidth]{Images/bell_curve_plot}
    \caption{Bell curves for numerical and analytical estimators, compared to the normal distribution.}
    \label{fig:Estimation-curves}
\end{figure}



% ----- Data Analysis -----
\section{Analysis}\label{sec:analysis}

\subsection{Testing Environment Parameters}\label{subsec:testing-environment-parameters}
For each sample size, 1000 repetitions were conducted, with \(\mu = 5\) being the true value for the simulations.


\subsection{Observations}\label{subsec:observations}
Both estimations generally follow the Law of Large Numbers, where the bias decreases as the sample size increases.
There is a slight difference between the two estimators at \(n = 50\) and \(n = 140\).
We also observe excess kurtosis close to 3 for all sample sizes, with values approaching 3 as the sample size increases.


\subsection{Discussion}\label{subsec:discussion}
The observed skewness and kurtosis are as expected, with kurtosis near 3, which is typical for the Bernoulli distribution.
As the sample size increases, the estimations become more similar to a normal distribution due to the Central Limit Theorem.
The higher the sample size, the closer the kurtosis gets to 3, indicating convergence towards normality.

Both estimators follow the Law of Large Numbers, showing a general decrease in bias.
However, at \(n = 140\), there is an unexpected increase in bias, which may be due to the randomness in the Monte Carlo simulation, where random numbers generated by the inverse function cause fluctuations.
Despite this, the overall trend supports the Law of Large Numbers, where bias decreases asymptotically as \(N\) approaches infinity, though it does so asymmetrically, leading to spikes and fluctuations.

There is a minor discrepancy between the estimators at \(n = 50\) and \(n = 140\), but this difference, occurring at the eighth decimal place, is negligible in the context of this experiment.
The small difference can be attributed to the nature of the methods used: the iterative algorithm behaves slightly differently from the algebraic estimation in the analytical method.
However, the difference in results between the numerical and analytical approaches is minimal.

\subsection{Conclusions}\label{subsec:conclusions}
\begin{itemize}
    \item Both estimators become more accurate as the sample size increases, following the Law of Large Numbers (LLN).
    \item The distribution assumes a normal distribution as the sample size increases, consistent with the Central Limit Theorem (CLT).
    \item Although the two estimators use different methods, their results are very similar, and the differences are negligible.
\end{itemize}

\end{document}
