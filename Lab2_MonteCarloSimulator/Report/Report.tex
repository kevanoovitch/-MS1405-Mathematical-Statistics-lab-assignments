\documentclass{report}
\usepackage[utf8]{inputenc}   % Encoding support
\usepackage{graphicx}         % To include images
\usepackage{listings}         % For code formatting
\usepackage{xcolor}           % To add colors
\usepackage{hyperref}         % For clickable links and references
\usepackage{amsmath, amssymb,amsthm} %For math
\usepackage{comment}
\usepackage{float}
\usepackage{minted}


\title{Report II}
\author{Kevin Deshayes}
\date{\2024-10-14}

\begin{document}

% ----- Title Page -----
\begin{titlepage}
  \centering
  {\huge\bfseries Report on Lab 2 \\[1cm]}  % Title
  \textbf{Author:} Kevin Deshayes\\[0.5cm]  % Author Name
  \textbf{Email:} kede23@student.bth.se\\[0.5cm]  % Email
  \textbf{Course Name:} Mathematical Statistics \\[0.5cm]  % Course Name
  \textbf{Course Code:} MS1403\\[1.5cm]  % Course Code

  \includegraphics[width=0.6\textwidth]{BTH_logo_gray.png} % Placeholder image
  \vfill
  \vspace{2cm}
\end{titlepage}

% ----- Table of Contents -----
\tableofcontents
\newpage
% ----- Code Example -----
\section{R code}



\begin{minted}[breaklines, linenos, frame=single, bgcolor=lightgray]{r}
#CODE HERE
\end{minted}

% ----- Results -----
\section{Results}
\begin{table}[htbp]
  \centering
  \begin{tabular}{|c|c|c|c|c|c|}
    \hline
    $n$ & Mean     & Bias        & MSE          & Skewness      & Kurtosis (excess) \\
    \hline
    10  & 4.708924 & -0.2910758  & 0.08472515   & -1.50566e-08  & 1                 \\
    20  & 4.981209 & -0.01879081 & 0.0003530945 & 0             & 1                 \\
    50  & 5.190263 & 0.190263    & 0.03620001   & 1.659551e-08  & 1                 \\
    90  & 5.078463 & 0.07846326  & 0.006156484  & -1.623733e-08 & 1                 \\
    140 & 5.088773 & 0.08877251  & 0.007880559  & -1.627026e-08 & 1                 \\
    \hline
  \end{tabular}
  \caption{Monte Carlo Simulation Results for Different Values of $n$}
\end{table}


\begin{figure}[H]
  \centering
  \includegraphics[width=1\linewidth]{plot.png}
  \caption{Generated Bias \& Mean plot}
  \label{fig:enter-label}
\end{figure}


% ----- Data Analysis -----
\section{Analysis}

\subsection{Observations}
Kurtosis is constant for all sample-sizes which is unexpected. Skewness in n = 20 is 0 indication that it's symmetric.
The bias decreases from n = 10 to n=20 but then flucturates at larger n.

The MSE decrease the larger the sample size but increases and spikes in n=50. There is a large decrease form n=10 to n=20.


\subsection{Discussion}
The reason for the constant value in kurtosis is because of the small amount of estimations. For each sample-size only two estimations are made (one numerical and one analytical).
This will lead to unmeaningfull kurtosis values which is why we don't see an increase or decrease in kurtosis when sample-size changes because estimations would also have to
increase in order to see changes. We can also see how the same limitations of the estimations affect the skewness. Though not constant the variation is very small
with skewness being 0 at n=20 which is an expected result when using such small sample-space and estimations because both theese values are veru susspetiable to sample-size and amount of estimations.

It is also possible to see how the estiamtions first underestimate on smaller sample sizes. But on larger sample sizes overestimates with the most accurate being
a sligh underestimation in n=20.

THE MSE decreases the larger the sample size which follow the law of large numbers. However, it spikes in n=50 this could be because of the sample size and
because of outliers in the estimations.






\subsection{Conclusions}

\begin{itemize}
  \item Gets more inaccurate the larger the sample size.
  \item The distirbutions follows overall the law of large numbers.
  \item The few estimations affect the meningfullness of Kurtosis and Skewness.
\end{itemize}

\end{document}
